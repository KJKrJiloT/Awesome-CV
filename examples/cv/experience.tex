%-------------------------------------------------------------------------------
%	SECTION TITLE
%-------------------------------------------------------------------------------
\cvsection{Research Experiences}


%-------------------------------------------------------------------------------
%	CONTENT
%-------------------------------------------------------------------------------
\begin{cventries}
%---------------------------------------------------------
  \cventry
    {POSTECH  Quantum Computing and Quantum Networks Lab.} % Job title
    {Master's candidate} % Organization
    {Pohang, Korea} % Location
    {Mar. 2023 - Present} % Date(s)
    {
      {\textbf{Outline: Construction of cryogenic ion trap system for quantum computing}}
      \vspace{\baselineskip}
      \begin{cvitems} % Description(s) of tasks/responsibilities
        \item {Investigated on nonlinear Duffing oscillator dynamics of $^{174}$Yb⁺ ion motion}
        \item {Designed and fabricated a multi-layer ion trap chip to implement a quantum computing platform}
        \item {Constructed an ultra-high vacuum (UHV) and electrical delivery system within a cryogenic station}
        \item {Implemented a 4 K cryogenic UHV system with EMCCD imaging, achieving successful detection of $^{40}$Ca⁺ ion fluorescence}
        \item {Designed and built a stable laser system and optical path for trapped-ion manipulation}
        \item {Developed a PyQt-based user interface(UI) to automate the experimental setup}
        \item {Trapped two $^{40}$Ca⁺ ion qubits and compensated micro-motion by reducing the Lorentzian linewidth in 397 nm spectroscopy}
        \item {Investigating how light-induced charging from oxide patches on trap electrodes distorts the trapping potential and drives ion-position drifts through a surface photovoltage (SPV)-based analysis (manuscript in preparation)}
        \item {Developing a real-time ion tracking program that employs a CNN model (Faster R-CNN) to automatically identify ion count and positions from EMCCD images (patent application in progress)}
      \end{cvitems}
    }
  \vspace{\baselineskip}

%---------------------------------------------------------
  \cventry
    {Hongik Univ. Advanced Semiconductor Technology Lab.} % Job title
    {Undergraduate student} % Organization
    {Seoul, Korea} % Location
    {Aug. 2021 - Dec. 2022} % Date(s)
    {
      {\textbf{Outline: Enhancement-mode operation of depletion-mode GaN HEMT by integrating with clamp circuit}}
      \vspace{\baselineskip}
      \begin{cvitems} % Description(s) of tasks/responsibilities
        \item {Modeled and analyzed AlGaN/GaN HEMT using TCAD(Silvaco Atlas)}
        \item {Converted TCAD model to Spice model by BSIM3 library}
        \item {Implemented a clamp circuit in LTspice to achieve normally-off operation and optimized switching speed}
        \item {Achieved high power conversion efficiency in a DC-DC boost converter}
      \end{cvitems}
    }
%---------------------------------------------------------
\end{cventries}
