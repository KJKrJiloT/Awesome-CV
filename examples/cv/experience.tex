%-------------------------------------------------------------------------------
%	SECTION TITLE
%-------------------------------------------------------------------------------
\cvsection{Research Experiences}


%-------------------------------------------------------------------------------
%	CONTENT
%-------------------------------------------------------------------------------
\begin{cventries}
%---------------------------------------------------------
  \cventry
    {POSTECH  Quantum Computing and Quantum Networks Lab.} % Job title
    {Master's candidate} % Organization
    {Pohang, Korea} % Location
    {Mar. 2023 - Present} % Date(s)
    {
      {\textbf{Outline: Construction of cryogenic ion trap system for quantum computing}}
      \vspace{\baselineskip}
      \begin{cvitems} % Description(s) of tasks/responsibilities
        \vspace{0.1 cm}
        \setlength{\itemsep}{0.2 em}
        \item {Investigated the \underline{nonlinear Duffing oscillator dynamics} of a single $^{174}$Yb⁺ ion}
        \item {Designed and fabricated a \underline{multi-layer ion trap chip} for scalable quantum-computing experiments}
        \item {Constructed an \underline{ultra-high vacuum (UHV)}, \underline{electrical delivery}, \underline{$^{40}$Ca⁺ ion fluorescence imaging system} within a 4 K cryogenic station}
        \item {Designed and built a \underline{stable laser system} and \underline{optical path} for trapped-ion manipulation}
        \item {Developed a PyQt-based \underline{user interface(UI)} to automate the experimental setup}
        \item {Trapped \underline{two $^{40}$Ca⁺ ion qubits} and compensated micro-motion by narrowing the Lorentzian linewidth in 397 nm spectroscopy}
        \item {Characterized \underline{light-induced charging effects} on trap electrodes and their impact on the ion-trapping potential}
        \item {Developing a \underline{real-time ion-detection program} based on a Faster R-CNN model to automatically infer ion number and positions from EMCCD images}
      \end{cvitems}
    }
  \vspace{\baselineskip}

%---------------------------------------------------------
  \cventry
    {Hongik Univ. Advanced Semiconductor Technology Lab.} % Job title
    {Undergraduate student} % Organization
    {Seoul, Korea} % Location
    {Aug. 2021 - Dec. 2022} % Date(s)
    {
      {\textbf{Outline: Enhancement-mode operation of depletion-mode GaN HEMT by integrating with clamp circuit}}
      \vspace{\baselineskip}
      \begin{cvitems} % Description(s) of tasks/responsibilities
        \vspace{0.1 cm}
        \setlength{\itemsep}{0.2 em}
        \item {Modeled and analyzed \underline{AlGaN/GaN HEMT} using TCAD(Silvaco Atlas)}
        \item {Converted TCAD model into \underline{Spice model by BSIM4 library}}
        \item {Integrated the HEMT model with a clamp circuit in LTspice to achieve normally-off operation and optimized switching speed}
        \item {Achieved \underline{high power conversion efficiency} in a GaN-based DC-DC boost converter}
      \end{cvitems}
    }
%---------------------------------------------------------
\end{cventries}
